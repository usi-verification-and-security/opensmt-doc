\documentclass{easychair}

\newcommand{\todo}[1]{$\langle\langle$#1 $\rangle\rangle$}
\usepackage{hyperref}

\newcommand{\qflra}{QF\_LRA}
\newcommand{\qflia}{QF\_LIA}
\newcommand{\qfuf}{QF\_UF}
\newcommand{\qfbv}{QF\_BV}
\newcommand{\qfrdl}{QF\_RDL}
\newcommand{\qfidl}{QF\_IDL}

\title{The OpenSMT Solver in SMT-COMP 2021 -- To Be Updated}
\author{
Martin Blicha \and 
Antti E. J. Hyv{\"a}rinen \and
Matteo Marescotti \and
Natasha Sharygina \\
}
\institute{Universit{\`a} della Svizzera italiana (USI), Lugano,
Switzerland}
\date{}
\titlerunning{The OpenSMT Solver in SMT-COMP 2021}
\authorrunning{M. Blicha, A. E. J. Hyv{\"a}rinen, M. Marescotti, and N.
Sharygina}
\begin{document}
\maketitle

\section{Overview}

OpenSMT~\cite{HyvarinenMAS16} is a T-DPLL based SMT
solver~\cite{NieuwenhuisOT:JACM06} that has been developed at USI,
Switzerland, since 2008.  The solver is written in {\tt C++} and
currently supports the quantifier-free logics of equality with
uninterpreted functions (\qfuf), linear real and integer arithmetic (\qflra, \qflia), and real and integer difference logics (\qfrdl, \qfidl).
OpenSMT also supports some aspects of bit-vector logic (\qfbv).

In comparison to 2020, the 2021 competition entry features a dedicated difference logic solver based on Exhaustive Theory Propagation \cite{Nieuwenhuis_2005} (contributed by V{\'a}clav Lu{\v n}{\'a}k) and a support for producing models for all fully supported logics (\qfuf, \qflra, \qflia, \qfrdl, \qfidl).
Additionally, the handling of ITE terms has been improved, resulting in a large performance benefit on \qflia{} benchmarks containing complex nested ITE terms.


OpenSMT features not exercised in the competition include support for a
wide range of interpolation algorithms for propositional
logic~\cite{AltFHS:VSTTE2015}, linear real
arithmetic~\cite{BlichaHKS19}, and uninterpreted
functions~\cite{AltHAS:FMCAD17} (available also in the incremental mode); an experimental look\-ahead-based
search algorithm~\cite{HyvarinenMSCS18} as an alternative to the more
standard CDCL algorithm; and features that support search-space
partitioning in particular designed for parallel
solving~\cite{HyvarinenMS:SAT15}.
OpenSMT is now also able to efficiently produce certificates of unsatisfiability~\cite{}\todo{\color{red}reference!\color{black}}, although this feature has not yet been merged to the main repository.

\section{External Code and Contributors}

The SAT solver driving OpenSMT is based on the MiniSAT
solver~\cite{EenS:SAT03}, and the rational number implementation is
inspired by a library written by David Monniaux.  Several people have
directly contributed to the OpenSMT code.  In alphabetical order, the
major contributors are
%
Leonardo Alt (Ethereum Foundation),
Sepideh Asadi (USI),
Martin Blicha (USI, Charles University),
Roberto Bruttomesso (Cybersecurity / Nozomi Networks),
Antti E. J. Hyv{\"a}rinen (USI),
V{\'a}clav Lu{\v n}{\'a}k (Charles University),
Matteo Marescotti (USI),
Rodrigo Benedito Otoni (USI),
Edgar Pek (University of Illinois, Urbana-Champaign),
Simone Fulvio Rollini (United Technologies Research Center),
Parvin Sadigova (King's College London), and
Aliaksei Tsitovich (Sonova).
%
The solver is being developed in Natasha Sharygina's software
verification group at USI.

\section{Utilization}

OpenSMT is used in a range of projects as a back-end solver.  It has been used as an interpolation engine of the Sally model
checker~\cite{JovanovicD:FMCAD16} which won the first and the second place in the transition systems category in the constrained Horn clause competition 2019 and 2020, respectively.
Recently, is has been used as the basis for a new CHC solver Golem which won the second place in LRA-TS and LIA-Lin categories in CHC-COMP 2021.
OpenSMT also forms the basis of our own model checkers such as
HiFrog~\cite{AltACMFHS17} and UpProver~\cite{Asadi_2020b}.
OpenSMT is compatible with the SMTS parallelization framework~\cite{MarescottiHS18}.

\section{Availability}
The source code repository and more information on the solver is
available at

\begin{itemize}
    \item \url{https://github.com/usi-verification-and-security/opensmt}
        and
    \item \url{http://verify.inf.usi.ch/opensmt}
\end{itemize}

\iffalse
in chronological order, work on interpolation 
algorithms~\cite{BlichaHKS19,AltHAS17,JancikAFHKS16,AsadiBFHESC18}
and parallel SMT 
solving~\cite{HyvarinenMSCS18,MarescottiHS18,HyvarinenMS:SAT15}.
OpenSMT2 is
used as the back-end in model-checking tools
HiFrog~\cite{AltACMFHS17},
eVolCheck~\cite{FSS_TACAS13}, 
FunFrog~\cite{SFS_ATVA12}, and
PeRIPLO~\cite{RolliniAFHS:LPAR2013,AltFHS:VSTTE2015}.
OpenSMT2 is a supported engine in the parallel 
solving framework SMTS~\cite{MarescottiHS16}.

\section{Acknowledgements}
We thank everyone who helped
developing OpenSMT2. In particular,
Leonardo Alt,
Sepideh Asadi,
Martin Blicha,
Roberto Bruttomesso,
Antti E. J. Hyv{\"a}rinen,
Matteo Marescotti,
Edgar Pek,
Simone Fulvio Rollini, 
Parvin Sadigova,
Natasha Sharygina,
Aliaksei Tsitovich.
\fi

\bibliography{abstract}
\bibliographystyle{plain}

\end{document}
